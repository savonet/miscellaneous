\documentclass{book}
\usepackage[utf8]{inputenc}
\usepackage[T1]{fontenc}
\usepackage{lmodern}
\usepackage{xspace}
\renewcommand{\familydefault}{\sfdefault}
\usepackage[papersize={5.5in,8.5in}]{geometry}
\usepackage{hyperref}

\newcommand{\prog}[1]{\emph{#1}}
\newcommand{\Liquidsoap}{\prog{Liquisoap}\xspace}
\newcommand{\eg}{e.g.~}

\author{David Baelde \and Romain Beauxis \and Samuel Mimram}
\title{Creating Net Radios Using Liquidsoap}
\hypersetup{
  % pdftitle={\csname @title\endcsname},
  % pdfauthor={\csname @author\endcsname},
  unicode=true,
  colorlinks=true,
  linkcolor=black,
  citecolor=black,
  urlcolor=black
}

\begin{document}
\maketitle
\tableofcontents
\chapter{Introduction}

\chapter{Quickstart}
\section{The Internet radio toolchain}
\Liquidsoap is a general audio stream generator, but is mainly intended for
Internet radios. Before starting with the proper Liquidsoap tutorial let's
describe quickly the components of the internet radio toolchain, in case the
reader is not familiar with it.

The chain is made of
\begin{itemize}
\item the stream generator (Liquidsoap,
  \href{http://www.icecast.org/ices.php}{ices}, or for example a DJ-software
  running on your local PC) which creates an audio stream (Ogg Vorbis or MP3);
\item the streaming media server (\href{http://www.icecast.org}{Icecast},
  \href{http://www.shoutcast.com}{Shoutcast}, ...) which relays several streams
  from their sources to their listeners;
\item the media player (xmms, Winamp, ...) which gets the audio stream from the
  streaming media server and plays it to the listener's speakers.
\end{itemize}

TODO image (Internet radio toolchain)

The stream is always passed from the stream generator to the server, whether or
not there are listeners. It is then sent by the server to every listener. The
more listeners you have, the more bandwidth you need.

If you use Icecast, you can broadcast more than one audio feed using the same
server. Each audio feed or stream is identified by its ``mount point'' on the
server. If you connect to the \verb+foo.ogg+ mount point, the URL of your stream
will be \href{http://localhost:8000/foo.ogg}{http://localhost:8000/foo.ogg} --
assuming that your Icecast is on localhost on port 8000. If you need further
information on this you might want to read Icecast's
\href{http://www.icecast.org}{documentation}. A proper setup of a streaming
server is required for running savonet.

Now, let's create an audio stream.

\section{Starting to use Liquidsoap}
In this tutorial we assume that you have a fully installedLiquidsoap. In
particular the library \verb+utils.liq+ should have been installed, otherwise
Liquidsoap won't know the operators which have been defined there. If you
installed into the default \verb+/usr/local+ you will find it inside
\verb+/usr/local/lib/liquidsoap/+.

\subsection{Sources}
A stream is built with Liquidsoap by using or creating sources. A source is an
annotated audio stream. In the following picture we represent a stream which has
at least three tracks (one of which starts before the snapshot), and a few
metadata packets (notice that they do not necessarily coincide with new tracks).

TODO image (A stream)In a Liquidsoap script, you build source
objects. Liquidsoap provides many functions for creating sources from scratch
(e.g. \verb+playlist+), and also for creating complex sources by putting
together simpler ones (e.g. \verb+switch+ in the following example). Some of
these functions (typically the \verb+output.*+) create an active source, which
will continuously pull its children's stream and output it to speakers, to a
file, to a streaming server, etc. These active sources are the roots of a
Liquidsoap instance, the sources which bring life into it.

\subsection{That source is fallible!}
A couple of things can go wrong in your streaming system.  In Liquidsoap, we say
that a source is \emph{infallible} if it will be always available.  Otherwise,
it is \emph{fallible}, something can go wrong.  By default, an output requires
that its input source is infallible, otherwise it complains that ``That source
is fallible!''

For example, a normal \verb+playlist+ will be fallible.  Firstly, because it
could contain only invalid files, or at least spend too much time on invalid
files for preparing a valid one on time.  Moreover, a playlist could contain
remote files, which may not be accessible quickly at all times.  A queue of user
requests is an other example of fallible source.

If \verb+file.ogg+ is a valid local file, then \verb+single("file.ogg")+ will be
an infallible source.  You can also build infallible playlists by using the
\verb+playlist.safe+ operator, which checks all files at startup, and won't
accept remote files -- but don't use it with too large playlists.

When an output complains about its source, you have to turn it into an
infallible one. Depending on the situation, many solutions are available. The
function \verb+mksafe+ takes a source and returns an infallible source,
streaming silence when the input stream becomes unavailable.  The default
speaker output \verb+out+ actually uses \verb+mksafe+ in its definition. In a
radio-like stream, silence is not the prefered solution, and you will probably
prefer to \verb+fallback+ on an infallible ``security'' source:

\begin{verbatim}
fallback([your_infallible_source_here, single("failure.ogg")])
\end{verbatim}
Finally, if you do not care about failures, you can pass the parameter
\verb+fallible=true+ to most outputs. In that case, the output
will accept a fallible source, and stop whenever the source fails,
to restart when it is ready to emit a stream again.
This is usually done if you are not emitting a radio-like stream,
but for example capturing or relaying another stream,
or encoding files.

\section{One-line expressions}
Liquidsoap is a scripting language. Many simple setups can be achieved by
evaluating one-line expressions.

\subsection{Playlists}
In the first example we'll play a playlist. Let's put a list of audio files in
\verb+playlist.pls+: one filename per line, lines starting with a \verb+#+ are
ignored. You can also put remote files' URLs, if your liquidsoap has
\href{help.html#plugins}{support} for the corresponding protocols.  Then just
run:

\begin{verbatim}
liquidsoap 'out(playlist("playlist.pls"))'
\end{verbatim}
Other playlist formats are supported, such as M3U and, depending on your
configuration, XSPF. Instead of giving the filename of a playlist, you can also
use a directory name, and liquidsoap will recursively look for audio files in
it.

Depending on your configuration, the output \verb+out+ will use AO, Alsa or OSS,
or won't do anything if you do not have support for these libs. In that case,
the next example is for you.

\subsection{Streaming out to a server}
Liquidsoap is capable of playing audio on your speakers, but it can also send
audio to a streaming server such as Icecast or Shoutcast. You can choose between
two widespread audio codecs: MP3 and Ogg Vorbis. One instance of liquidsoap can
stream one audio feed in many formats (and even many audio feeds in many
formats!).

You may already have an Icecast server. Otherwise you can install and configure
your own Icecast server. The configuration typically consists in setting the
admin and source passwords, in \verb+/etc/icecast2/icecast.xml+. These passwords
should really be changed if your server is visible from the hostile internet,
unless you want people to kick your source as admins, or add their own source
and steal your bandwidth.

We are now going to send an audio stream, encoded as Ogg Vorbis, to an Icecast
server:
\begin{verbatim}
liquidsoap
  'output.icecast(%vorbis,
     host = "localhost", port = 8000, \
     password = "hackme", mount = "liq.ogg", \
     mksafe(playlist("playlist.m3u")))'
\end{verbatim}
The main difference with the previous is that we used
\verb+output.icecast.vorbis+ instead of \verb+out+. The second difference is the
use of the \verb+mksafe+ which turns your fallible playlist source into an
infallible source.

Streaming to Shoutcast is quite similar, using the \verb+output.shoutcast+
function:

\begin{verbatim}
liquidsoap 'output.shoutcast(%mp3, \
                host="localhost", port = 8000, \
	        password = "changeme", \
	        mksafe(playlist("playlist.m3u")))'
\end{verbatim}

\subsection{Input from another streaming server}
Liquidsoap can use another stream as an audio source. This may be useful if you
do some live shows.

\begin{verbatim}
liquidsoap '
  out(input.http("http://dolebrai.net:8000/dolebrai.ogg"))'
\end{verbatim}
\subsection{Input from the soundcard}
If you're lucky and have a working ALSA support, try one of these... but beware
that ALSA may not work out of the box.

\begin{verbatim}
liquidsoap 'output.alsa(input.alsa())'
\end{verbatim}
\begin{verbatim}
liquidsoap 'output.alsa(bufferize = false,
                        input.alsa(bufferize = false))'
\end{verbatim}
\subsection{Other examples}
You can play with many more examples. Here are a few more. To build your own,
lookup the \href{reference.html}{API documentation} to check what functions are
available, and what parameters they accept.

\begin{verbatim}
# Listen to your playlist, but normalize the volume
liquidsoap 'out(normalize(playlist("playlist_file")))'
\end{verbatim}
\begin{verbatim}
# ... same, but also add smart cross-fading
liquidsoap 'out(smart_crossfade(
                  normalize(playlist("playlist_file"))))'
\end{verbatim}
\section{Script files}
We have seen how to create a very basic stream using one-line expressions. If
you need something a little bit more complicated, they will prove uneasy to
manage. In order to make your code more readable, you can write it down to a
file, named with the extension \verb+.liq+ (eg: \verb+myscript.liq+).

To run the script:

\begin{verbatim}

liquidsoap myscript.liq
\end{verbatim}
On UNIX, you can also put \verb+#!/path/to/your/liquidsoap+ as the first line of
your script (``shebang''). Don't forget to make the file executable:

\begin{verbatim}

chmod u+x myscript.liq
\end{verbatim}
Then you'll be able to run it like this:

\begin{verbatim}

./myscript.liq
\end{verbatim}
Usually, the path of the liquidsoap executable is \verb+#/usr/bin/liquidsoap+,
and we'll use this in the following.

\section{A simple radio}
We will start with a basic radio station, that plays songs randomly chosen from
a playlist, adds a few jingles (more or less one every four songs), and output
an Ogg Vorbis stream to an Icecast server.

Before reading the code of the corresponding liquidsoap script, it might be
useful to visualize the streaming process with the following tree-like
diagram. The idea is that the audio streams flows through this diagram,
following the arrows. In this case the nodes (\verb+fallback+ and \verb+random+)
select one of the incoming streams and relay it. The final node
\verb+output.icecast+ is an output: it actively pulls the data out of the graph
and sends it to the world.

TODO image (Graph for 'basic-radio.liq')

\begin{verbatim}
#!/usr/bin/liquidsoap
# Log dir
set("log.file.path","/tmp/basic-radio.log")

# Music
myplaylist = playlist("~/radio/music.m3u")
# Some jingles
jingles = playlist("~/radio/jingles.m3u")
# If something goes wrong, we'll play this
security = single("~/radio/sounds/default.ogg")

# Start building the feed with music
radio = myplaylist
# Now add some jingles
radio = random(weights = [1, 4],[jingles, radio])
# And finally the security
radio = fallback(track_sensitive = false, [radio, security])

 # Stream it out
output.icecast(%vorbis,
  host = "localhost", port = 8000,
  password = "hackme", mount = "basic-radio.ogg",
  radio)
\end{verbatim}

\subsection{What's next?}
You can first have a look at a \href{complete_case.html}{more complex
  example}. There is also a second tutorial about \href{advanced.html}{advanced
  techniques}.

You should definitely learn \href{help.html}{how to get help}.  If you know
enough liquidsoap for your use, you'll only need to refer to the
\href{reference.html}{scripting reference}, or see the
\href{cookbook.html}{cookbook}.  At some point, you might read more about
Liquidsoap's \href{language.html}{scripting language}.  For a better
understanding of liquidsoap, it is also useful to read a bit about the notions
of \href{sources.html}{sources} and \href{requests.html}{requests}.
\end{document}
