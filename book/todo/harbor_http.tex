\section{Harbor as HTTP server}
The harbor server can be used as a HTTP server. You 
can use the function \verb+harbor.http.register+ to register
HTTP handlers for GET and POST requests. Its parameters
are are follow:

\begin{verbatim}
harbor.http.register(port=8080,uri,handler)
\end{verbatim}
 where:

\begin{itemize}
\item \verb+port+ is the port where to receive incoming connections
\item \verb+uri+ is used to match requested uri. Perl regular expressions are accepted.
\item \verb+handler+ is the function used to process requests.

\end{itemize}
\verb+handler+ function has type:

\begin{verbatim}
(~method:string, ~protocol:string, ~data:string, 
 ~headers:[(string*string)], string)->string))->unit
\end{verbatim}
where:

\begin{itemize}
\item \verb+method+ is the HTTP method used by the client. Currently, one of \verb+"POST"+ or \verb+"GET"+
\item \verb+protocol+ is the HTTP protocol used by the client. Currently, one of \verb+"HTTP/1.0"+ or \verb+"HTTP/1.1"+
\item \verb+data+ is the data passed during a POST request
\item \verb+headers+ is the list of HTTP headers sent by the client
\item \verb+string+ is the (unparsed) uri requested by the client, e.g.: \verb+"/foo?var=bar"+

\end{itemize}
The \verb+handler+ function returns HTTP and HTML data to be sent to the client,
for instance:

\begin{verbatim}
HTTP/1.1 200 OK\r\n\
Content-type: text/html\r\n\
Content-Length: 35\r\n\
\r\n\
<html><body>It works!</body></html>
\end{verbatim}
(\verb+\r\n+ should always be used for line return
in HTTP content)

For convenience, a \verb+http_response+ function is provided to 
create a HTTP response string. It has the following type:

\begin{verbatim}
(?protocol:string,?code:int,?headers:[(string*string)],
 ?data:string)->string
\end{verbatim}
where:

\begin{itemize}
\item \verb+protocol+ is the HTTP protocol of the response (default \verb+HTTP/1.1+)
\item \verb+code+ is the response code (default \verb+200+)
\item \verb+headers+ is the response headers. It defaults to \verb+[]+ but an appropriate \verb+"Content-Length"+ header is added if not set by the user and \verb+data+ is not empty.
\item \verb+data+ is an optional response data (default \verb+""+)

\end{itemize}
Thess functions can be used to create your own HTTP interface. Some examples
are:

\subsection{Redirect Icecast's pages}
Some source clients using the harbor may also request pages that
are served by an icecast server, for instance listeners statistics.
In this case, you can register the following handler:

\begin{verbatim}
# Redirect all files other
# than /admin.* to icecast,
# located at localhost:8000
def redirect_icecast(~method,~protocol,~data,~headers,uri) =
   http_response(
     protocol=protocol,
     code=301,
     headers=[("Location","http://localhost:8000#{uri}")]
   )
end

# Register this handler at port 8005
# (provided harbor sources are also served
#  from this port).
harbor.http.register(port=8005,"^/(?!admin)",
                     redirect_icecast)
\end{verbatim}
Another alternative, less recommanded, is to
directly fetch the page's content from the Icecast server:

\begin{verbatim}
# Serve all files other
# than /admin.* by fetching data
# from Icecast, located at localhost:8000
def proxy_icecast(~method,~protocol,~data,~headers,uri) =
  def f(x) =
    # Replace Host
    if string.capitalize(fst(x)) == "HOST" then
      "Host: localhost:8000"
    else
      "#{fst(x)}: #{snd(x)}"
    end
  end
  headers = list.map(f,headers)
  headers = string.concat(separator="\r\n",headers)
  request = 
    "#{method} #{uri} #{protocol}\r\n\
     #{headers}\r\n\r\n"
  get_process_output("echo #{quote(request)} | \
                      nc localhost 8000")
end

# Register this handler at port 8005
# (provided harbor sources are also served
#  from this port).
harbor.http.register(port=8005,"^/(?!admin)",
                     proxy_icecast)
\end{verbatim}
This method is not recommenced because some servers may not
close the socket after serving a request, causing \verb+nc+ and
liquidsoap to hang.

\subsection{Get metadata}
You can use harbor to register HTTP services to 
fecth/set the metadata of a source. For instance, 
using the \href{json.html}{JSON export function} \verb+json_of+:

\begin{verbatim}
meta = ref []

# s = some source

# Update current metadata
# converted in UTF8
def update_meta(m) =
  m = metadata.export(m)
  recode = string.recode(out_enc="UTF-8")
  def f(x) =
    (recode(fst(x)),recode(snd(x)))
  end
  meta := list.map(f,m)
end

# Apply update_metadata
# every time we see a new
# metadata
s = on_metadata(update_meta,s)

# Return the json content
# of meta
def get_meta(~method,~protocol,~data,~headers,uri) =
  m = !meta
  http_response(
    protocol=protocol,
    code=200,
    headers=[("Content-Type","application/json; charset=utf-8")],
    data=json_of(m)
  )
end

# Register get_meta at port 700
harbor.http.register(port=7000,"/getmeta",get_meta)
\end{verbatim}
Once the script is running, 
a GET/POST request for \verb+/getmeta+ at port \verb+7000+
returns the following:

\begin{verbatim}
HTTP/1.1 200 OK
Content-Type: application/json; charset=utf-8

{
  "genre": "Soul",
  "album": "The Complete Stax-Volt Singles: 1959-1968 (Disc 8)",
  "artist": "Astors",
  "title": "Daddy Didn't Tell Me"
}
\end{verbatim}
Which can be used with AJAX-based backends to fetch the current 
metadata of source \verb+s+

\subsection{Set metadata}
Using \verb+insert_metadata+, you can register a GET handler that
updates the metadata of a given source. For instance:

\begin{verbatim}

# s = some source

# x is of type ((metadata)->unit)*source
# first part is a function used to update
# metadata and second part is the source 
# whose metadata are updated
x = insert_metadata(s)

# Get the function
insert = fst(x)

# Redefine s as the new source
s = snd(x)

# The handler
def set_meta(~method,~protocol,~data,~headers,uri) =
  # Split uri of the form request?foo=bar&...
  # into (request,[("foo","bar"),..])
  x = url.split(uri)

  # Filter out unusual metadata
  meta = metadata.export(snd(x))
  
  # Grab the returned message
  ret =
    if meta != [] then
      insert(meta)
      "OK!"
    else
      "No metadata to add!"
  end

  # Return response
  http_response(
   protocol=protocol,
   code=200,
   headers=[("Content-Type","text/html")],
   data="<html><body><b>#{ret}</b></body></html>"
  )
end

# Register handler on port 700
harbor.http.register(port=7000,"/setmeta",set_meta)
\end{verbatim}
Now, a request of the form \verb+http://server:7000/setmeta?title=foo+
will update the metadata of source \verb+s+ with \verb+[("title","foo")]+. You
can use this handler, for instance, in a custom HTML form.

\section{Limitations}
When using harbor's HTTP server, please be warned that the server is 
\textbf{not} meant to be used under heavy load. Therefore, it should \textbf{not}
be exposed to your users/listeners if you expect many of them. In this
case, you should use it as a backend/middle-end and have some kind of 
caching between harbor and the final user. In particular, the harbor server
is not meant to server big files because it loads their entire content in 
memory before sending them. However, the harbor HTTP server is fully equipped 
to serve any kind of CGI script. 

