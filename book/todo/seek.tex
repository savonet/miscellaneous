\section{Seeking in liquidsoap}
Starting with Liquidsoap \verb+1.0.0-beta2+, it is now possible to seek within sources! 
Not all sources support seeking though: currently, they are mostly file-based sources
such as \verb+request.queue+, \verb+playlist+, \verb+request.dynamic+ etc..

The basic function to seek within a source is \verb+source.seek+. It has the following type:

\begin{verbatim}
(source('a),float)->float
\end{verbatim}
The parameters are:

\begin{itemize}
\item The source to seek.
\item The duration in seconds to seek from current position.

\end{itemize}
The function returns the duration actually seeked.

Please note that seeking is done to a position relative to the \emph{current}
position. For instance, \verb+source.seek(s,3.)+ will seek 3 seconds forward in
source \verb+s+ and \verb+source.seek(s,(-4.))+ will seek 4 seconds backward.

Since seeking is currently only supported by request-based sources, it is recommended
to hook the function as close as possible to the original source. Here is an example
that implements a server/telnet seek function:

\begin{verbatim}
# A playlist source
s = playlist("/path/to/music")

# The server seeking function
def seek(t) =
  t = float_of_string(default=0.,t)
  log("Seeking #{t} sec")
  ret = source.seek(s,t)
  "Seeked #{ret} seconds."
end

# Register the function
server.register(namespace=source.id(s),
                description="Seek to a relative position \
                             in source #{source.id(s)}",
                usage="seek <duration>",
                "seek",seek)
\end{verbatim}
\section{Cue points.}
Sources that support seeking can also be used to implement cue points.
The basic operator for this is \verb+cue_cut+. Its has type:

\begin{verbatim}
(?id:string,?cue_in_metadata:string,
 ?cue_out_metadata:string,
 source(audio='#a,video='#b,midi='#c))->
    source(audio='#a,video='#b,midi='#c)
\end{verbatim}
Its parameters are:

\begin{itemize}
\item \verb+cue_in_metadata+: Metadata for cue in points, default: \verb+"liq_cue_in"+.
\item \verb+cue_out_metadata+: Metadata for cue out points, default: \verb+"liq_cue_out"+.
\item The source to apply cue points to.

\end{itemize}
The values of cue-in and cue-out points are given in absolute
position through the source's metadata. For instance, the following
source will cue-in at 10 seconds and cue-out at 45 seconds on all its tracks:

\begin{verbatim}
s = playlist(prefix="annotate:liq_cue_in=\"10.\",liq_cue_out=\"45\":",
             "/path/to/music")

s = cue_cut(s)
\end{verbatim}
As in the above example, you may use the \verb+annotate+ protocol to pass custom cue
points along with the files passed to Liquidsoap. This is particularly useful 
in combination with \verb+request.dymanic+ as an external script can build-up
the appropriate URI, including cue-points, based on information from your
own scheduling back-end.

Alternatively, you may use \verb+map_metadata+ to add those metadata. The operator
\verb+map_metadata+ supports seeking and passes it to its underlying source.

