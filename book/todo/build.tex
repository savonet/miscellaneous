\section{Building liquidsoap from source}
From every sub-project's directory (ocaml-vorbis, ocaml-dtools, liquidsoap, 
etc.) you can build using \verb+./bootstrap+, \verb+./configure+, 
\verb+make+ and optionally install using \verb+make install+.

If you are using the \verb+liquidsoap-full+ tarball,
or the
\href{https://savonet.svn.sourceforge.net/svnroot/savonet/trunk}{SVN trunk},
there is also a fast procedure for building liquidsoap
that doesn't require installing the libraries provided by Savonet.
The steps to follow are simple.

First,
you should choose which features you want to enable when building liquidsoap.
Each shipped feature can be enabled/disabled by editing the
\verb+PACKAGE+ file.
Depending on your version you might have to first copy 
\verb+PACKAGES.default+ to \verb+PACKAGES+.

Then run the usual commands from the toplevel directory, \emph{above} the
sub-project's directories:

\begin{verbatim}
# Edit PACKAGES to choose which feature you want

# In the SVN tree you should bootstrap first.
# No need with the tarballs.
./bootstrap

# Configure all libraries and packages.
# You may pass extra options such as --enable-debugging,
# --prefix, --sysconfdir, --localstatedir, etc.
./configure

# Now, build all libraries and liquidsoap
make

# To install liquidsoap,
# you'll usually need to type the following as root
make install
\end{verbatim}
\subsection{Dependencies}
Here are liquidsoap's dependencies (all OCaml libraries are distributed by Savonet, except when linked):

\begin{itemize}
\item \href{http://www.ocaml-programming.de/programming/findlib.html}{ocamlfind}
\item ocaml-dtools
\item ocaml-duppy (>= 0.1.2)
\item \href{http://www.ocaml.info/home/ocaml_sources.html}{ocaml-pcre}

\end{itemize}
And also optional dependencies. For most of these, you also need
the associated C/C++ library, which is usually provided by your distribution's
packaging system.

\begin{itemize}
\item ocaml-ogg for ogg audio and video formats
\item ocaml-vorbis for ogg/vorbis audio encoding and decoding
\item ocaml-cry for Icecast/Shoutcast streaming
\item ocaml-mad for mp3 decoding
\item ocaml-lame for mp3 encoding
\item ocaml-faac for AAC encoding
\item ocaml-faad for AAC stream decoding
\item ocaml-aacplus for AAC+ encoding
\item ocaml-flac for native flac and ogg/flac encoding/decoding
\item ocaml-theora for ogg/theora encoding/decoding
\item ocaml-schroedinger for ogg/dirac video encoding/decoding
\item ocaml-gavl for video conversion
\item ocaml-samplerate for audio samplerate conversion
\item ocaml-taglib for MP3 audio tag reading
\item ocaml-magic for file type detection. This library is very useful for reading mp3 files. If not enabled, liquidsoap may accept as mp3 files that do not contains mp3 data, such as jpeg pictures.
\item ocaml-xmlplaylist for XML-based playlist formats
\item ocaml-soundtouch for soundtouch audio effects
\item ocaml-lastfm and ocaml-xmllight for lastfm protocol support
\item \href{http://camomile.sourceforge.net/}{camomile} for detecting metadata encoding and re-encoding them to utf8
\item ocaml-alsa for ALSA input/output
\item ocaml-ao for AO I/O
\item ocaml-portaudio for Portaudio I/O
\item ocaml-bjack for Jack output/input
\item ocaml-ladspa for LADSPA plugins
\item \href{http://erratique.ch/software/xmlm}{xmlm}

\end{itemize}
Runtime dependencies include:

\begin{itemize}
\item \href{http://www.gnu.org/software/wget/}{wget} for downloading remote files (http, https, ftp)
\item ufetch (provided by ocaml-fetch) for downloading remote files (smb, http, ftp)
\item \href{http://www.cstr.ed.ac.uk/projects/festival/}{festival} for speech synthesis (say)

\end{itemize}
And other that you'll find on the project page, or in liquidsoap-full tarball.

