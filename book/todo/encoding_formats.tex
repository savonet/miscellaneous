Since after version 0.9.3, liquidsoap has decoding formats. These are
special values describing how to encode raw data.
Practically, this means that instead of writing:

\begin{verbatim}
output.icecast.vorbis(quality=0.3,samplerate=44100,...)
\end{verbatim}
you shall now write:

\begin{verbatim}
output.icecast(%vorbis(quality=0.3,samplerate=44100),etc)
\end{verbatim}
The same goes for \verb+output.file+ and for other formats.

\section{List of formats and their syntax}
All parameters are optional, and the parenthesis are not needed
when no parameter is passed. In the following default values
are shown.
As a special case, the keywords \verb+mono+ and \verb+stereo+ can be used to indicate
the number of channels (whether is is passed as an integer or a boolean).

\subsection{MP3}
\begin{verbatim}
%mp3(stereo=true, samplerate=44100, bitrate=128, quality=5, id3v2=false)
\end{verbatim}
\begin{itemize}
\item Only one of \verb+bitrate+ or \verb+quality+ should be specified.
\item \verb+id3v2=true+ is only valid if liquidsoap has been compiled with taglib support.

\end{itemize}
\subsection{WAV}
\begin{verbatim}
%wav(stereo=true, channels=2, samplesize=16, header=true, duration=10.)
\end{verbatim}
If \verb+header+ is \verb+false+, the encoder outputs raw PCM. \verb+duration+ is optional
and is used to set the WAV length header.

Because Liquidsoap encodes a possibly infinite stream, there
is no way to know in advance the duration of encoded data. Since WAV header
has to be written first, by default its length is set to the maximun possible 
value. If you know the expected duration of the encoded data and you actually 
care about the WAV length header then you should use this parameter.

\subsection{Ogg}
The following formats can be put together in an Ogg container.
The syntax for doing so is \verb+%ogg(x,y,z)+ but it is also
possible to just write \verb+%vorbis(...)+, for example, instead
of \verb+%ogg(%vorbis(...))+.

\subsubsection{Vorbis}
\begin{verbatim}
# Variable bitrate
%vorbis(samplerate=44100, channels=2, quality=0.3)
% Average bitrate
%vorbis.abr(samplerate=44100, channels=2, bitrate=128, max_bitrate=192, min_bitrate=64)
# Constant bitrate
%vorbis.cbr(samplerate=44100, channels=2, bitrate=128)
\end{verbatim}
Quality ranges from -0.2 to 1,
but quality -0.2 is only available with the aotuv implementation of libvorbis.

\subsubsection{Theora}
\begin{verbatim}
%theora(quality=40,width=w,height=h,
        picture_width=w,picture_height=h,
        picture_x=0, picture_y=0,
        aspect_numerator=1, aspect_denominator=1,
        keyframe_frequency=64, vp3_compatible=false,
        soft_target=false, buffer_delay=0.1,
        speed=0)
\end{verbatim}
You can also pass \verb+bitrate=x+ explicitly instead of a quality.
The default dimensions are liquidsoap's default,
from the settings \verb+frame.video.height/width+.

\subsubsection{Dirac}
\begin{verbatim}
%dirac(quality=35,width=w,height=h,
       picture_x=0, picture_y=0,
       aspect_numerator=1, aspect_denominator=1)
\end{verbatim}
\subsubsection{Speex}
\begin{verbatim}
%speex(stereo=false, samplerate=44100, quality=7,
       mode=[wideband|narrowband|ultra-wideband],
       frames_per_packet=1,
       complexity=none)
\end{verbatim}
You can also control quality using \verb+abr=x+ or \verb+vbr=y+.

\subsubsection{Flac}
The flac encoding format comes in two flavors:

\begin{itemize}
\item \verb+%flac+ is the native flac format, useful for file output but not for streaming purpose
\item \verb+%ogg(%flac,...)+ is the ogg/flac format, which can be used to broadcast data with icecast

\end{itemize}
The parameters are:

\begin{verbatim}
%flac(samplerate=44100, 
      channels=2, 
      compression=5, 
      bits_per_sample=16)
\end{verbatim}
\verb+compression+ ranges from 0 to 8 and \verb+bits_per_sample+ should be one of: \verb+8+, \verb+16+ or \verb+32+.

\subsection{AAC}
The syntax for the internal AAC encoder is:

\begin{verbatim}
%aac(channels=2, samplerate=44100, bitrate=64, adts=true)
\end{verbatim}
\subsection{AAC+}
The syntax for the internal AAC+ encoder is:

\begin{verbatim}
%aacplus(channels=2, samplerate=44100, bitrate=64)
\end{verbatim}
\subsection{External encoders}
For a detailed presentation of external encoders, see \href{external_encoders.html}{this page}.

\begin{verbatim}
%external(channels=2,samplerate=44100,header=true,
          restart_on_crash=false,
          restart_on_new_track,
          restart_after_delay=<int>,
          process="")
\end{verbatim}
Only one of \verb+restart_on_new_track+ and \verb+restart_after_delay+ should
be passed. The delay is specified in seconds.
The encoding process is mandatory, and can also be passed directly
as a string, without \verb+process=+.

\section{Formats determine the stream content}
In most liquidsoap scripts, the encoding format determines what
kind of data is streamed.

The type of an encoding format depends on its parameter.
For example, \verb+%mp3+ has type \verb+format(audio=2,video=0,midi=0)+
but \verb+%mp3(mono)+ has type \verb+format(audio=1,video=0,midi=0)+.

The type of an output like \verb+output.icecast+
or \verb+output.file+ is something like
\begin{verbatim}
(...,format('a),...,source('a))->source('a)
\end{verbatim}
.
This means that your source will have to have the same type as your format.

For example if you write
\begin{verbatim}
output.file(%mp3,"/tmp/foo.mp3",playlist("~/audio"))
\end{verbatim}

then the playlist source will have to stream stereo audio.
Thus it will reject mono and video files.

\section{Technical details}
You can store an atomic format in a variable, it is a value like another:
\verb+fmt = %mp3+. However, an atomic format is an atomic constant despite its
appearance. You cannot use a variable for one of its parameters: for
example 

\begin{verbatim}
x = 44100
%vorbis(samplerate=x)
\end{verbatim}
is not allowed,
you must write \verb+%vorbis(samplerate=44100)+.

In programming languages like ML, the typing of \verb+printf+ is a bit special.
Alone, \verb+printf+ has an esoteric type. Together with its parameter, it
takes a meaningful type, for example \verb+printf "An integer: %d\n"+ has type
\verb+int -> unit+. So, the format string \verb+"An integer: %d\n"+ is not a string
at all, it has a more complex type, and cannot be manipulated as a string.
Our encoding formats have a similar role, hence the symbol \verb+%+.

